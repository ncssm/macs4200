\documentclass[12pt]{amsart}
\usepackage{amssymb,amsmath,amsthm,graphicx,verbatim,amsbsy}

\usepackage[margin=1in]{geometry}
\newtheorem{theorem}{Theorem}[section]
\newtheorem{lemma}[theorem]{Lemma}
\newtheorem{corollary}[theorem]{Corollary}
\newtheorem{proposition}[theorem]{Proposition}
\newtheorem{noname}[theorem]{}
\newtheorem{sublemma}{}[theorem]
\newtheorem{conjecture}[theorem]{Conjecture}

\theoremstyle{definition}
\newtheorem{definition}[theorem]{Definition}
\newtheorem{example}[theorem]{Example}

\theoremstyle{remark}
\newtheorem{remark}[theorem]{Remark}

\numberwithin{equation}{section}

\begin{document}

\begin{flushright}
Name:\_\_\_\_\_\_\_\_\_\_\_\_\_\_\_\_\_\_\_\_\_\_\_
\end{flushright}
\vspace{10pt}
\begin{center}
Introduction to Cryptography

Introduction to the Euclidean Algorithm
\end{center}


\begin{enumerate}
\item Without using a computer, use the Euclidean algorithm to compute the greatest common divisor of the following pairs of integers. How many steps does each computation take?
\begin{enumerate}
\item $gcd(468,864)$
\item $gcd(11111,111111)$
\end{enumerate}

\item Let $n$ be a positive integer. Use the Euclidean Algorithm to compute $gcd(n+1,n)$.

\item We know that every algorithm must terminate in a finite number of steps. Explain why the Euclidean algorithm is guaranteed to terminate in a finite number of steps.

\item Without using a computer, use the Euclidean algorithm to compute the greatest common divisor of the following pairs of Fibonacci numbers. How many steps does each computation take? 
\begin{enumerate}
\item $gcd(F_3,F_2)$
\item $gcd(F_4,F_3)$
\item $gcd(F_5,F_4)$
\item $gcd(F_6,F_5)$
\end{enumerate}

\item Make a conjecture regarding the number of steps needed to compute $gcd(F_{n+1},F_n)$ using the Euclidean algorithm.

\item Let $a$ and $b$ be two positive integers such that $a>b$. Suppose it takes exactly $n$ steps to compute $gcd(a,b)$ using the Euclidean algorithm. This means that the $gcd(a,b)$ can be obtained from the following steps.
\begin{align*}
a&=bq_1+r_1 \text{ where } 0\leq r_1< b\text{ and } q_1\geq 1\\
b&=r_1q_2+r_2 \text{ where } 0\leq r_2< r_1\text{ and } q_2\geq 1\\
r_1&=r_2q_3+r_3 \text{ where } 0\leq r_3< r_2\text{ and } q_3\geq 1\\
\vdots\\
r_{n-5}&=r_{n-4}q_{n-3}+r_{n-3} \text{ where } 0\leq r_{n-3}< r_{n-4}\text{ and } q_{n-3}\geq 1\\
r_{n-4}&=r_{n-3}q_{n-2}+r_{n-2} \text{ where } 0\leq r_{n-2}< r_{n-3}\text{ and } q_{n-2}\geq 1\\
r_{n-3}&=r_{n-2}q_{n-1}+r_{n-1} \text{ where } 0\leq r_{n-1}< r_{n-2}\text{ and } q_{n-1}\geq 1\\
r_{n-2}&=r_{n-1}q_{n}+0 \text{ where } q_n\geq 1
\end{align*}
\begin{enumerate}
\item Explain why $r_{n-1}\geq 1$. (Hint: Why can't we have that $r_{n-1}=0$?)
\item Show why $r_{n-2}\geq 1$.
\item Show why $r_{n-3}\geq 2$.
\item Show why $r_{n-4}\geq 3$.
\item Show why $r_{n-5}\geq 5$.
\item Make a conjecture about the smallest possible value of $b$.
\item Make a conjecture about the smallest possible value of $a$.
\end{enumerate}

\item Using your conjectures above, what can be said about the number of digits in $a$ and $b$ if it takes $100$ steps to compute $gcd(a,b)$ using the Euclidean algorithm?

\item Let $a$ and $b$ be $100$ digit positive integers such that $a>b$. At most how many steps would it take to compute $gcd(a,b)$ using the Euclidean algorithm?

\end{enumerate}
\vfill


\end{document}