\documentclass[12pt]{amsart}
\usepackage{amssymb,amsmath,amsthm,graphicx,verbatim,amsbsy}

\usepackage[margin=1in]{geometry}
\newtheorem{theorem}{Theorem}[section]
\newtheorem{lemma}[theorem]{Lemma}
\newtheorem{corollary}[theorem]{Corollary}
\newtheorem{proposition}[theorem]{Proposition}
\newtheorem{noname}[theorem]{}
\newtheorem{sublemma}{}[theorem]
\newtheorem{conjecture}[theorem]{Conjecture}

\theoremstyle{definition}
\newtheorem{definition}[theorem]{Definition}
\newtheorem{example}[theorem]{Example}

\theoremstyle{remark}
\newtheorem{remark}[theorem]{Remark}

\numberwithin{equation}{section}

\begin{document}

\begin{flushright}
Name:\_\_\_\_\_\_\_\_\_\_\_\_\_\_\_\_\_\_\_\_\_\_\_
\end{flushright}
\vspace{10pt}
\begin{center}
Introduction to Cryptography

Euclidean Algorithm Program
\end{center}


\begin{enumerate}
\item Write a function in Python which 
\begin{enumerate}
\item accepts as arguments two integers $a$ and $b$;
\item returns a list whose first element is the $gcd(a,b)$, whose second element is the number of iterations needed in the Euclidean algorithm, and whose third element is the time taken to run the iterations. 
\end{enumerate} 
\item Use your function to find the greatest common divisor of the pairs of numbers below. Record the computer run time and iteration count for each set.
\begin{enumerate}
\item $a=135301852344706746049, b=947112966412947222343$
\item $a=354224848179261915075, b=573147844013817084101$
\item $a=573147844013817084101, b=927372692143078999176$
\end{enumerate}
\item Write a function in Python which 
\begin{enumerate}
\item accepts a positive integer $n$ as an argument;
\item returns a list of all positive integers which are less than $n$ and relatively prime to $n$.
\end{enumerate}
\item Euler's totient function is the function $\phi(n)$ which gives the number of positive integers less than $n$ which are relatively prime to $n$. Use your function in the previous problem to find the following values.
\begin{enumerate}
\item $\phi(2)$
\item $\phi(3)$
\item $\phi(5)$
\item $\phi(7)$
\item $\phi(11)$
\item $\phi(6)$
\item $\phi(10)$
\item $\phi(14)$
\item $\phi(15)$
\item $\phi(21)$
\item $\phi(33)$
\item $\phi(35)$
\end{enumerate}
\item Conjecture a formula to compute $\phi(p)$ where $p$ is a prime number.
\item Conjecture a formula to compute $\phi(pq)$ where $p$ and $q$ are distinct prime numbers.
\end{enumerate}
\vfill


\end{document}