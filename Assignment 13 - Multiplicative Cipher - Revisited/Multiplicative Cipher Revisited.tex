\documentclass[12pt]{amsart}
\usepackage{amssymb,amsmath,amsthm,graphicx,verbatim,amsbsy}

\usepackage[margin=1in]{geometry}
\newtheorem{theorem}{Theorem}[section]
\newtheorem{lemma}[theorem]{Lemma}
\newtheorem{corollary}[theorem]{Corollary}
\newtheorem{proposition}[theorem]{Proposition}
\newtheorem{noname}[theorem]{}
\newtheorem{sublemma}{}[theorem]
\newtheorem{conjecture}[theorem]{Conjecture}

\theoremstyle{definition}
\newtheorem{definition}[theorem]{Definition}
\newtheorem{example}[theorem]{Example}

\theoremstyle{remark}
\newtheorem{remark}[theorem]{Remark}

\numberwithin{equation}{section}

\begin{document}

\begin{flushright}
Name:\_\_\_\_\_\_\_\_\_\_\_\_\_\_\_\_\_\_\_\_\_\_\_
\end{flushright}
\vspace{10pt}
\begin{center}
Introduction to Cryptography

Multiplicative Cipher - Revisited
\end{center}


\begin{enumerate}
\item Write a function in Python that 
\begin{enumerate}
\item accepts a lowercase message (string) and a multiplicative key (integer) as arguments;
\item checks that the key is valid (mod $26$);
\item performs the multiplicative encryption to each letter modulo $26$;
\item returns the encrypted message.
\end{enumerate}
\item Write a function in Python that 
\begin{enumerate}
\item accepts a plaintext integer, multiplicative key (integer), and a modulus $n$ (integer) as arguments;
\item checks that the key is valid (mod $n$);
\item checks that the modulus is larger than the plaintext integer;
\item performs the multiplicative encryption to the integer modulo $n$;
\item prints the encrypted integer.
\end{enumerate}
\item By hand, encrypt the number $14$ with a modulus of $13$ and a multiplicative key of $2$. By hand, encrypt the number $1$ with a modulus of $13$ and a multiplicative key of $2$. What happens if the modulus is less than or equal to the plaintext integer?
\item Use your function from problem 2 to encrypt the number $1234567$ by multiplying by the key $A=319765$ using the modulus $27989898$.
\item Let $A$ be an integer. A {\bf multiplicative inverse of $A$ mod $n$} is an integer $x$ such that $Ax\equiv 1 \mod n$. 
\begin{enumerate}
\item Let $A=2$. Test each value of $x$ where $0\leq x \leq 25$ and determine whether there exists a multiplicative inverse of $A$ modulo $26$ (please don't do this manually).
\item Repeat the above exercise for each value of $A$ where $0\leq A\leq 25$ (please don't do this manually).
\item Make a conjecture regarding which values of $A$ have a multiplicative inverse modulo $26$.
\item Make a conjecture regarding which values of $A$ have a multiplicative inverse modulo $n$.
\end{enumerate}
\item Verify that the multiplicative inverse of $A=319765$ is $27698239$ modulo $27989898$. How can this be used to decrypt your answer in problem 4?
\end{enumerate}
\vfill


\end{document}