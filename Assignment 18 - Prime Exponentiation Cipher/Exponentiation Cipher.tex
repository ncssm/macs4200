\documentclass[12pt]{amsart}
\usepackage{amssymb,amsmath,amsthm,graphicx,verbatim,amsbsy}
\usepackage[margin=1in]{geometry}

\newtheorem{theorem}{Theorem}[section]
\newtheorem{lemma}[theorem]{Lemma}
\newtheorem{corollary}[theorem]{Corollary}
\newtheorem{proposition}[theorem]{Proposition}
\newtheorem{noname}[theorem]{}
\newtheorem{sublemma}{}[theorem]
\newtheorem{conjecture}[theorem]{Conjecture}

\theoremstyle{definition}
\newtheorem{definition}[theorem]{Definition}
\newtheorem{example}[theorem]{Example}

\theoremstyle{remark}
\newtheorem{remark}[theorem]{Remark}

\numberwithin{equation}{section}

\begin{document}

\begin{flushright}
Name:\_\_\_\_\_\_\_\_\_\_\_\_\_\_\_\_\_\_\_\_\_\_\_
\end{flushright}
\vspace{10pt}
\begin{center}
Introduction to Cryptography

The Repeated Squaring Algorithm and the Prime Exponentiation Cipher
\end{center}



\begin{enumerate}
\item Using Python, write a function which
\begin{enumerate}
\item accepts integers $a$, $b$, and $n$ as arguments;
\item implements the repeated squaring algorithm without using the pow(a,b,n) function;
\item returns the value of $a^b\mod n$.
\end{enumerate}

\item Using Python, write a function which
\begin{enumerate}
\item accepts an integer message, exponential key (integer), a modulus $p$ (prime integer), and a boolean variable as arguments;
\item checks that the key is valid;
\item performs the exponential encryption to the integer message modulo $p$ if the boolean variable is TRUE and performs the exponential decryption to the integer message modulo $p$ if the boolean variable is FALSE;
\item prints the encrypted/decrypted integer message.
\end{enumerate}
\item Suppose we publicly broadcast the modulus $p$ and the encryption exponent $e$ of the prime exponentiation cipher so that anyone can encrypt a message. This is called a {\bf public key encryption} scheme. If the decryption exponent $d$ is kept private (and not publicly broadcast), explain how an attacker would still be able to decrypt an intercepted ciphertext message.
\item Suppose we intercept the following ciphertext message. $$265447333455441482929244853322644679466$$ This message was encrypted using the public key encryption scheme outlined in the previous problem where the prime modulus is 

\noindent $p=1111111112223333333344455566777777779999$ and the encryption exponent is $e=19088892923$. Decrypt the message.
\end{enumerate}


\end{document}