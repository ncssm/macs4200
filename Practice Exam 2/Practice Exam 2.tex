% Exam Template for UMTYMP and Math Department courses
%
% Using Philip Hirschhorn's exam.cls: http://www-math.mit.edu/~psh/#ExamCls
%
% run pdflatex on a finished exam at least three times to do the grading table on front page.
%
%%%%%%%%%%%%%%%%%%%%%%%%%%%%%%%%%%%%%%%%%%%%%%%%%%%%%%%%%%%%%%%%%%%%%%%%%%%%%%%%%%%%%%%%%%%%%%

% These lines can probably stay unchanged, although you can remove the last
% two packages if you're not making pictures with tikz.
\documentclass[11pt]{exam}
\RequirePackage{amssymb, amsfonts, amsmath, latexsym, verbatim, xspace, setspace}
\RequirePackage{tikz, pgflibraryplotmarks}

% By default LaTeX uses large margins.  This doesn't work well on exams; problems
% end up in the "middle" of the page, reducing the amount of space for students
% to work on them.
\usepackage[margin=1in]{geometry}


% Here's where you edit the Class, Exam, Date, etc.
\newcommand{\class}{Cryptography}
\newcommand{\term}{Spring 2022}
\newcommand{\examnum}{Practice Exam 2}
\newcommand{\examdate}{March 30, 2022}
\newcommand{\timelimit}{50 Minutes}

% For an exam, single spacing is most appropriate
\singlespacing
% \onehalfspacing
% \doublespacing

% For an exam, we generally want to turn off paragraph indentation
\parindent 0ex

\begin{document}

% These commands set up the running header on the top of the exam pages
\pagestyle{head}
\firstpageheader{}{}{}
\runningheader{\class}{\examnum\ - Page \thepage\ of \numpages}{\examdate}
\runningheadrule

\begin{flushright}
\begin{tabular}{p{2.8in} r l}
\textbf{\class} & \textbf{Name (Print):} & \makebox[2in]{\hrulefill}\\
\textbf{\term} &&\\
\textbf{\examnum} &&\\
\textbf{\examdate} &&\\
\textbf{Time Limit: \timelimit} & Section & \makebox[2in]{\hrulefill}
\end{tabular}\\
\end{flushright}
\rule[1ex]{\textwidth}{.1pt}


This exam contains \numpages\ pages (including this cover page) and
\numquestions\ problems.  Check to see if any pages are missing.  Enter
all requested information on the top of this page and sign 
the Honor Code pledge at the bottom of this page.\\

You are required to show your work on each problem on this exam.  The following 
rules apply:\\

\begin{minipage}[t]{3.7in}
\vspace{0pt}
\begin{itemize}

\item \textbf{Organize your work} in a reasonably neat and coherent way. Clearly label your work and circle your answer. Work scattered all over the page without a clear ordering will receive very little credit.

\item This is an open notes exam, but you may not copy code from any online source. 

\item Ask for scratch paper if you need more space. Clearly label any work completed on scratch paper.
\end{itemize}

Do not write in the table to the right.

\vspace{.5 in}

\textbf{Honor Code Pledge: }By signing below, you are verifying that you have completed this examination in accordance with the ethical standards expected at NCSSM.

\vspace{.5 in}

\textbf{Signature:} \makebox[2in]{\hrulefill}\\

\end{minipage}
\hfill
\begin{minipage}[t]{2.3in}
\vspace{0pt}
%\cellwidth{3em}
\gradetablestretch{2}
\vqword{Problem}
\addpoints % required here by exam.cls, even though questions haven't started yet.
\gradetable[v]%[pages]  % Use [pages] to have grading table by page instead of question

\end{minipage}
\newpage % End of cover page

%%%%%%%%%%%%%%%%%%%%%%%%%%%%%%%%%%%%%%%%%%%%%%%%%%%%%%%%%%%%%%%%%%%%%%%%%%%%%%%%%%%%%
%
% See http://www-math.mit.edu/~psh/#ExamCls for full documentation, but the questions
% below give an idea of how to write questions [with parts] and have the points
% tracked automatically on the cover page.
%
%
%%%%%%%%%%%%%%%%%%%%%%%%%%%%%%%%%%%%%%%%%%%%%%%%%%%%%%%%%%%%%%%%%%%%%%%%%%%%%%%%%%%%%

\begin{questions}


%Limit Calculations
\newpage
\addpoints
\question 
\begin{parts}
\part[10] Write a function in Python which
\begin{enumerate}
\item accepts no arguments;
\item uses a while loop to prompt the user for a single number in each loop, and stops looping once the user types ``stop";
\item returns the average of all numbers typed by the user.
\end{enumerate}
\part[10] Suppose you want to encrypt a message by reversing the order of all characters in the message. As an example, "I LOVE MATHEMATICS" would be encrypted as "SCITAMEHTAM EVOL I". Write a function in Python which 
\begin{enumerate}
\item accepts a plaintext message (string);
\item reverses the order of all characters in the message;
\item prints the encrypted message
\end{enumerate}
\end{parts}
\question
\begin{parts}
\part[5] Suppose you want to encrypt an integer message using a multiplicative cipher with modulus of $n$ where $n$ is larger than the integer message. How many valid multiplicative keys are there? 
\part[5] Explain how we know that $25$ has a multiplicative inverse modulo $109$, and then find the multiplicative inverse of $25$ modulo $109$ without using a calculator or computer. You must show all work.
\part[5] Explain how multiplicative inverses are used to decrypt integer messages in the multiplicative cipher, and then decrypt the integer $69$ which was encrypted by multiplying a plaintext integer by $25$ modulo $109$. You must show all work.
\end{parts}
\end{questions}
\end{document}
