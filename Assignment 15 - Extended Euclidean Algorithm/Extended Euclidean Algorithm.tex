\documentclass[12pt]{amsart}
\usepackage{amssymb,amsmath,amsthm,graphicx,verbatim,amsbsy}
\usepackage[margin=1in]{geometry}

\newtheorem{theorem}{Theorem}[section]
\newtheorem{lemma}[theorem]{Lemma}
\newtheorem{corollary}[theorem]{Corollary}
\newtheorem{proposition}[theorem]{Proposition}
\newtheorem{noname}[theorem]{}
\newtheorem{sublemma}{}[theorem]
\newtheorem{conjecture}[theorem]{Conjecture}

\theoremstyle{definition}
\newtheorem{definition}[theorem]{Definition}
\newtheorem{example}[theorem]{Example}

\theoremstyle{remark}
\newtheorem{remark}[theorem]{Remark}

\numberwithin{equation}{section}

\begin{document}

\begin{flushright}
Name:\_\_\_\_\_\_\_\_\_\_\_\_\_\_\_\_\_\_\_\_\_\_\_
\end{flushright}
\vspace{10pt}
\begin{center}
Introduction to Cryptography

The Extended Euclidean Algorithm
\end{center}



\begin{enumerate}
\item Write a function in Python which 
\begin{enumerate}
\item accepts as arguments three integers $a$, $b$, and $n$;
\item prints an error message if there is no solution to the Diophantine equation $AX+BY=n$;
\item returns a list of the form $\left[X,Y,count,time\right]$ where the pair $(X,Y)$ is a solution to the equation $AX+BY=n$, the third element is the number of iterations needed in the Euclidean algorithm, and the fourth element is the time taken to run the iterations. 
\end{enumerate} 
\item Use your function from the previous problem to solve the Diophantine equation $aX+bY=n$ for each set of integers below. Record a solution, computer run time and iteration count for each triple. 
\begin{enumerate}
\item $a=13259581529781261112802, b=1894225932825894444686, n=35$
\item $a=354224848179261915075, b=573147844013817084101, n=5$
\item $a=573147844013817084101, b=927372692143078999176, n= 21$
\end{enumerate}

\item Write a function in Python which 
\begin{enumerate}
\item accepts as arguments two integers $A$ and $n$;
\item returns the multiplicative inverse of $A \mod n$ if such a number exists, and returns FALSE otherwise.
\end{enumerate} 

\end{enumerate}


\end{document}