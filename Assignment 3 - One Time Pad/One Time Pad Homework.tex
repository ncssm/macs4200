\documentclass[12pt]{amsart}
\usepackage{amssymb,amsmath,amsthm,graphicx,verbatim,amsbsy}

\usepackage[margin=1in]{geometry}
\newtheorem{theorem}{Theorem}[section]
\newtheorem{lemma}[theorem]{Lemma}
\newtheorem{corollary}[theorem]{Corollary}
\newtheorem{proposition}[theorem]{Proposition}
\newtheorem{noname}[theorem]{}
\newtheorem{sublemma}{}[theorem]
\newtheorem{conjecture}[theorem]{Conjecture}

\theoremstyle{definition}
\newtheorem{definition}[theorem]{Definition}
\newtheorem{example}[theorem]{Example}

\theoremstyle{remark}
\newtheorem{remark}[theorem]{Remark}

\numberwithin{equation}{section}

\begin{document}

\begin{flushright}
Name:\_\_\_\_\_\_\_\_\_\_\_\_\_\_\_\_\_\_\_\_\_\_\_
\end{flushright}
\vspace{10pt}
\begin{center}
Introduction to Cryptography

One-Time Pad
\end{center}

Answer the following questions.

\begin{enumerate}
\item A one-time pad is an encryption technique that adds a different key to each letter. For example, if the plain text message that we wish to encrypt is $DOLPHIN$, and we use a one time pad of $XYHSNET$, then we compute $D+X$, $O+Y$, $L+H$, $P+S$, $H+N$, $I+E$, and $N+T$ to obtain an encrypted message $AMSHUMG$. The one-time pad must be at least as long as the message.

Implement a program in Python that
\begin{enumerate}
\item takes a plaintext message in a string;
\item takes a one-time pad in a string;
\item prints the encrypted ciphertext obtained by adding each letter of the plaintext message to the corresponding letter in the one-time pad.
\end{enumerate}
\end{enumerate}
\vfill


\end{document}