% Exam Template for UMTYMP and Math Department courses
%
% Using Philip Hirschhorn's exam.cls: http://www-math.mit.edu/~psh/#ExamCls
%
% run pdflatex on a finished exam at least three times to do the grading table on front page.
%
%%%%%%%%%%%%%%%%%%%%%%%%%%%%%%%%%%%%%%%%%%%%%%%%%%%%%%%%%%%%%%%%%%%%%%%%%%%%%%%%%%%%%%%%%%%%%%

% These lines can probably stay unchanged, although you can remove the last
% two packages if you're not making pictures with tikz.
\documentclass[11pt]{exam}
\RequirePackage{amssymb, amsfonts, amsmath, latexsym, verbatim, xspace, setspace}
\RequirePackage{tikz, pgflibraryplotmarks}

% By default LaTeX uses large margins.  This doesn't work well on exams; problems
% end up in the "middle" of the page, reducing the amount of space for students
% to work on them.
\usepackage[margin=1in]{geometry}


% Here's where you edit the Class, Exam, Date, etc.
\newcommand{\class}{Cryptography}
\newcommand{\term}{Spring 2022}
\newcommand{\examnum}{Practice Exam 1}
\newcommand{\examdate}{February 23, 2022}
\newcommand{\timelimit}{50 Minutes}

% For an exam, single spacing is most appropriate
\singlespacing
% \onehalfspacing
% \doublespacing

% For an exam, we generally want to turn off paragraph indentation
\parindent 0ex

\begin{document}

% These commands set up the running header on the top of the exam pages
\pagestyle{head}
\firstpageheader{}{}{}
\runningheader{\class}{\examnum\ - Page \thepage\ of \numpages}{\examdate}
\runningheadrule

\begin{flushright}
\begin{tabular}{p{2.8in} r l}
\textbf{\class} & \textbf{Name (Print):} & \makebox[2in]{\hrulefill}\\
\textbf{\term} &&\\
\textbf{\examnum} &&\\
\textbf{\examdate} &&\\
\textbf{Time Limit: \timelimit} & Section & \makebox[2in]{\hrulefill}
\end{tabular}\\
\end{flushright}
\rule[1ex]{\textwidth}{.1pt}


This exam contains \numpages\ pages (including this cover page) and
\numquestions\ problems.  Check to see if any pages are missing.  Enter
all requested information on the top of this page and sign 
the Honor Code pledge at the bottom of this page.\\

You are required to show your work on each problem on this exam.  The following 
rules apply:\\

\begin{minipage}[t]{3.7in}
\vspace{0pt}
\begin{itemize}

\item \textbf{Organize your work} in a reasonably neat and coherent way. Clearly label your work and circle your answer. Work scattered all over the page without a clear ordering will receive very little credit.

\item This is an open notes exam, but you may not copy code from any online source. 

\item Ask for scratch paper if you need more space. Clearly label any work completed on scratch paper.
\end{itemize}

Do not write in the table to the right.

\vspace{.5 in}

\textbf{Honor Code Pledge: }By signing below, you are verifying that you have completed this examination in accordance with the ethical standards expected at NCSSM.

\vspace{.5 in}

\textbf{Signature:} \makebox[2in]{\hrulefill}\\

\end{minipage}
\hfill
\begin{minipage}[t]{2.3in}
\vspace{0pt}
%\cellwidth{3em}
\gradetablestretch{2}
\vqword{Problem}
\addpoints % required here by exam.cls, even though questions haven't started yet.
\gradetable[v]%[pages]  % Use [pages] to have grading table by page instead of question

\end{minipage}
\newpage % End of cover page

%%%%%%%%%%%%%%%%%%%%%%%%%%%%%%%%%%%%%%%%%%%%%%%%%%%%%%%%%%%%%%%%%%%%%%%%%%%%%%%%%%%%%
%
% See http://www-math.mit.edu/~psh/#ExamCls for full documentation, but the questions
% below give an idea of how to write questions [with parts] and have the points
% tracked automatically on the cover page.
%
%
%%%%%%%%%%%%%%%%%%%%%%%%%%%%%%%%%%%%%%%%%%%%%%%%%%%%%%%%%%%%%%%%%%%%%%%%%%%%%%%%%%%%%

\begin{questions}


%Limit Calculations
\newpage
\addpoints
\question 
\begin{parts}
\part[10] Write a function in Python which accepts integers $a$, $b$, and $n$ as inputs and returns the sum of all the (distinct) multiples of $a$ or $b$ which are less that $n$.
\part[10] Write a function in Python which prompts the user for an integer $k$ as an input and returns a list of all distinct prime factors of $k$.
\part[10] (Bonus) Let $n$ be a positive integer. The Collatz sequence is defined recursively by the following rules:

\begin{align*}
A_0&=n\\
A_{i+1}&=\begin{cases} 
      \frac{A_{i}}{2} & \text{if $A_i$ is even} \\
      3A_i+1 & \text{if $A_i$ is odd} 
   \end{cases}
\end{align*}

Using the rule above and starting with $13$, we generate the following sequence:

$$13, 40, 20, 10, 5, 16, 8, 4, 2, 1$$

It can be seen that this sequence (starting at $13$ and finishing at $1$) contains 10 terms. The Collatz Conjecture states that all Collatz sequences terminate at $1$ in finitely many steps.

Which starting number, under one million, produces the longest Collatz sequence?
\end{parts}
\question Write a short response outlining the strengths and weaknesses of each of the following encryption schemes.
\begin{parts}
\part[5] Additive Caesar Cipher
\part[5] One-Time Pad
\end{parts}
\question Determine whether the following statements are true or false, in general. If a statement is true, show why. If a statement is false, explain why or provide a counterexample.
\begin{parts}
\part[5] $(x+y)^2\equiv x^2+y^2\mod 2$
\part[5] If $x+z\equiv y+z \mod n$, then $x\equiv y\mod n$
\part[5] The last digit of $7^{20222023}$ is $1$. 
\end{parts}
\end{questions}
\end{document}
