\documentclass[12pt]{amsart}
\usepackage{amssymb,amsmath,amsthm,graphicx,verbatim,amsbsy}

\usepackage[margin=1in]{geometry}
\newtheorem{theorem}{Theorem}[section]
\newtheorem{lemma}[theorem]{Lemma}
\newtheorem{corollary}[theorem]{Corollary}
\newtheorem{proposition}[theorem]{Proposition}
\newtheorem{noname}[theorem]{}
\newtheorem{sublemma}{}[theorem]
\newtheorem{conjecture}[theorem]{Conjecture}

\theoremstyle{definition}
\newtheorem{definition}[theorem]{Definition}
\newtheorem{example}[theorem]{Example}

\theoremstyle{remark}
\newtheorem{remark}[theorem]{Remark}

\numberwithin{equation}{section}

\begin{document}

\begin{flushright}
Name:\_\_\_\_\_\_\_\_\_\_\_\_\_\_\_\_\_\_\_\_\_\_\_
\end{flushright}
\vspace{10pt}
\begin{center}
Introduction to Cryptography

Fibonacci Numbers
\end{center}


\begin{enumerate}
\item Fibonacci Numbers are the sequence defined recursively by 
\begin{align*}
F_0&=1\\
F_1&=1\\
F_n&=F_{n-1}+F_{n-2} \text{ where } n\geq2
\end{align*}
\begin{enumerate}
\item Write a function in Python that accepts an integer $n$ as an input and returns the value of $F_n$.
\item Using your Fibonacci number generator, demonstrate that $$\lim_{n\rightarrow\infty} \frac{F_n}{F_{n-1}}=\frac{1+\sqrt{5}}{2}$$

Note: $\frac{1+\sqrt{5}}{2}$ is also called the Golden Ratio.
\item Write a program to make a list of all prime Fibonacci numbers less than one million.
\item Write a function in Python that returns the number of digits in an integer $n$. Use your function to find the smallest Fibonacci number with $1000$ digits.
\item Compute $gcd(F_3,F_2)$, $gcd(F_4,F_3)$, $gcd(F_5, F_4)$, and $gcd(F_6,F_5)$. Make a conjecture for the value of $gcd(F_{n+1},F_n)$.
\end{enumerate}

\end{enumerate}
\vfill


\end{document}