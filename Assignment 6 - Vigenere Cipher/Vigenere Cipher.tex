\documentclass[12pt]{amsart}
\usepackage{amssymb,amsmath,amsthm,graphicx,verbatim,amsbsy}

\usepackage[margin=1in]{geometry}
\newtheorem{theorem}{Theorem}[section]
\newtheorem{lemma}[theorem]{Lemma}
\newtheorem{corollary}[theorem]{Corollary}
\newtheorem{proposition}[theorem]{Proposition}
\newtheorem{noname}[theorem]{}
\newtheorem{sublemma}{}[theorem]
\newtheorem{conjecture}[theorem]{Conjecture}

\theoremstyle{definition}
\newtheorem{definition}[theorem]{Definition}
\newtheorem{example}[theorem]{Example}

\theoremstyle{remark}
\newtheorem{remark}[theorem]{Remark}

\numberwithin{equation}{section}

\begin{document}

\begin{flushright}
Name:\_\_\_\_\_\_\_\_\_\_\_\_\_\_\_\_\_\_\_\_\_\_\_
\end{flushright}
\vspace{10pt}
\begin{center}
Introduction to Cryptography

Vigenere Cipher
\end{center}


\begin{enumerate}
\item If a one-time pad is shorter than the message, one can simply repeat the pad until the message is complete. The method is called a Vigenere cipher. For example, we can encrypt the plain text message ILOVEMATHEMATICS using the Vigenere key CAT by computing ``ILOVEMATHEMATICS"+``CATCATCATCATCATC" to obtain ``KLHXEFCTAGMTVIVU".

Implement a program in Python that 
\begin{enumerate}
\item Prompts the user whether to encrypt or decrypt a message;
\item prompts the user for the plaintext or ciphertext message;
\item prompts the user for an appropriate key and implements the Vigenere cipher if the key is shorter than the message and implements the one-time pad if the key is longer than the message;
\item skips over punctuation and spaces whenever performing the encryption/decryption;
\item prints the encrypted or decrypted message as a string.
\end{enumerate}

\end{enumerate}
\vfill


\end{document}